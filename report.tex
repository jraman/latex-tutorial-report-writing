\documentclass{article}

\usepackage[T1]{fontenc}    % to get small-caps with \textsc
% \usepackage{bold-extra}       % better bold fonts

% loren ipsum ...
\usepackage{lipsum}

% Margin control.  includefoot => margin includes footer.
\usepackage[margin=1.0in,left=1.5in,includefoot]{geometry}

% Header and footer
\usepackage{fancyhdr}
\pagestyle{fancy}
\fancyhead{}    % turn off default header
\fancyfoot{}    % turn off default footer
\fancyfoot[R]{\thepage}     % page-number flushed right
\renewcommand{\headrulewidth}{0pt}  % turn off top (head) rule
\renewcommand{\footrulewidth}{0pt}  % turn off bottom (foot) rule
% end - header and footer

% Graphics control
\usepackage{graphicx}   % import images
\usepackage{float}      % control positions of floating objects

% Add hyperlinks to references, including in the TOC
\usepackage[hidelinks]{hyperref}

% Do not hyphenate and break words in a table
\usepackage[none]{hyphenat}

% Table
% https://tex.stackexchange.com/questions/257128/how-does-the-newcolumntype-command-work
% https://texfaq.org/FAQ-wholerow
% http://mirror.las.iastate.edu/tex-archive/macros/latex/required/tools/array.pdf
\usepackage{array}
\newcolumntype{$}{>{\global\let\currentrowstyle\relax}}
\newcolumntype{^}{>{\currentrowstyle}}
\newcommand{\rowstyle}[1]{\gdef\currentrowstyle{#1}%
  #1\ignorespaces
}
\newcommand{\PreserveBackslash}[1]{\let\temp=\\#1\let\\=\temp}
\newcolumntype{C}[1]{>{\PreserveBackslash\centering}p{#1}}
\newcolumntype{R}[1]{>{\PreserveBackslash\raggedleft}p{#1}}
\newcolumntype{L}[1]{>{\PreserveBackslash\raggedright}p{#1}}

% Bibliography
% sort the numbered references, and compress with number ranges.
\usepackage[numbers,sort&compress]{natbib}

\begin{document}

\begin{titlepage}

\begin{center}
% "\\" starts a new paragraph - can also use double-return instead.
% \line(slope, intercept){length}
\line(1, 0){300}\\
[0.25in]
\huge\bfseries My Awesome Report\\
[2mm]
\line(1, 0){200}\\
[1.5cm]
\textsc{\LARGE Author Name}\\
[0.75cm]
\textsc{\Large Using LaTeX to Write a Simple Report}\\
[9cm]
\begin{flushright}
\textsc{\large Leslie L.\\
A LaTeX User\\
\# 123456789\\
19 March 2019\\}
\end{flushright}

\end{center}

\end{titlepage}

% change front matter page numbering to roman (different from arabic for rest of doc)
\pagenumbering{roman}
% "*" excludes this section from being numbered
\section*{Acknowledgments}
% \numberline ensures that Acknowledgments is aligned with the text in the TOC, and
% not the section numbers.
\addcontentsline{toc}{section}{\numberline{}Acknowledgments}
I am grateful and thankful for all the people in my life who have helped me.

\cleardoublepage

% Probably needs more than one pass to build the TOC
\tableofcontents
\thispagestyle{empty}    % remove page-number on TOC page
\cleardoublepage
\setcounter{page}{1}    % skip TOC page numbering

% List of figures
\listoffigures
\addcontentsline{toc}{section}{\numberline{}List of Figures}
\cleardoublepage

% List of tables
\listoftables
\addcontentsline{toc}{section}{\numberline{}List of Tables}
\cleardoublepage

\section{Introduction}\label{sec:intro}
This is the first line of the report.  Below, we include a placeholder text in latin which is
a jumbled form of text by Cicero.

\lipsum[1]

% Insert a figure
\begin{figure}[H]
    \centering
    \includegraphics[height=3in]{figures/smiling-marmot.jpg}
    \caption[Optional caption for list of figures]{A smiling marmot.  This is the caption that goes under the figure.}
    \label{fig:smiling-marmot}
\end{figure}

In Fig.~\ref{fig:smiling-marmot}, you see a smiling marmot.  Marmots are large squirrels in the
genus Marmota, with 15 species.  A groundhog, woodchuck, or whistlepig, of the species \textit{M. monax}
is found in most of North America.


% start next section on a new page
\newpage
\section{Groundhog Lifestyle}
In this section we discuss the groundhog's lifestyle.  As you may know, groundhogs live in a hole
in the ground and also climb trees.

In the introduction on \pageref{sec:intro}, we did not say much.  But this is how a page reference
works.

\subsection{Diet}
Mostly herbivorous, groundhogs eat primarily wild grasses and other vegetation, including berries
and agricultural crops, when available.  An adult groundhog will eat more than a pound of vegetation daily.

% Table
\begin{table}[H]    % H is for here
    \centering
    \caption[Optional table caption for table of contents]{Potentially longer caption for the table itself.
    \label{tab:species}
    Genus Marmota.  Subgenus Marmota.}
    \begin{tabular}{l c r}    % alignment: left, center, right
    Species & Latin name & Habitat\\ \hline
    Alaskan marmot & M. broweri & Alaska\\
    Alpine marmot & M. marmota & Only in Europe including the Alps\\
    Black-capped marmot & M. camtschatica & Siberia\\
    \end{tabular}
\end{table}

Table~\ref{tab:species} shows some of the species in the genus Marmota.

% Used in conjuction with package array.  Also, specify the width of the columns.
% http://www.btaa.org/docs/default-source/research-data/at-a-glance_2016btaa.pdf?sfvrsn=4e7c4cf3_2
% https://tex.stackexchange.com/questions/286950/how-to-create-a-table-with-fixed-column-widths
\begin{table}[t]
    \centering
    \caption[Complicated table]{This is a complicated table.}
    % specify column width
    % \begin{tabular}{$p{0.4in}^p{2in}^p{1.0in}^p{1.0in}^p{1.2in}}
    % align and specify column width
    \begin{tabular}{$R{0.4in}^L{2in}^R{1.0in}^R{1.0in}^R{1.2in}}
    \rowstyle{\bfseries} Sl. No. & Name & Year Founded & Year Joined & Student Population \\ \hline
    1. & University of Illinois & 1867 & 1896 & 42,453 \\
    2. & Indiana University & 1820 & 1899 & 38,398 \\
    3. & University of Iowa & 1847 & 1899 & 25,624 \\
    5. & University of Michigan & 1817 & 1896 & 42,361 \\
    6. & Michigan State University & 1855 & 1950 & 43,695 \\
    7. & University of Minnesota & 1851 & 1896 & 39,302 \\
    9. & Northwestern University & 1851 & 1896 & 17,475 \\
    10. & Ohio State University & 1870 & 1912 & 51,817 \\
    12. & Purdue University & 1869 & 1896 & 35,884 \\
    14. & University of Wisconsin-Madison & 1848 & 1896 & 38,284 \\ \hline
    \multicolumn{5}{c}{\bfseries{Most Recently Added}} \\ \hline
    4. & University of Maryland & 1856 & 2014 & 34,444 \\
    8. & University of Nebraska-Lincoln & 1869 & 2011 & 22,062 \\
    11. & Pennsylvania State University & 1855 & 1990 & 45,874 \\
    13. & Rutgers University & 1766 & 2014 & 42,749 \\
    \end{tabular}
\end{table}

\subsubsection{Omnivores}
Groundhogs also occasionally eat grubs, grasshoppers, insects, snails and other small animals,
but are not as omnivorous as many other Sciuridae.  Groundhogs will occasionally eat baby birds
they come upon by accident.

\cleardoublepage

\section{Citations}
The TeXbook\cite{Knuth84} is a seminal work - this is also an example citation.  Leslie Lamport worked on LaTeX
while Knuth worked on TeX.  Here is an example of natbib sorting the citations\cite{Lamp86,Knuth84}.
The book by Mittelbach and Goossens\cite{Mitt04} has also been influential.
Here is an example of natbib hyphenating the citations\cite{Knuth84,Lamp86,Mitt04}.

\cleardoublepage

% \addcontentsline{toc}{section}{\numberline{}References}
\bibliographystyle{IEEEtran}
% ref.bib was created with BibDesk
\bibliography{references/ref}

\end{document}


% Notes
% * By default, \textsc produced the warning:
%   Font shape `OT1/cmr/bx/sc' undefined (Font) using `OT1/cmr/bx/n' instead on input line ...
%   Added: \usepackage[T1]{fontenc}
%   See:
%   https://tex.stackexchange.com/questions/37618/when-using-textsc-latex-issues-warning-font-shape-ot1-cmr-bx-sc-undefined
%   https://tex.stackexchange.com/questions/27411/small-caps-and-bold-face
% * The spacing directives, e.g. [9cm], need to be specified after the double backslash (\\).
% 